%iffalse
\let\negmedspace\undefined
\let\negthickspace\undefined
\documentclass[journal,12pt,twocolumn]{IEEEtran}
\usepackage{cite}
\usepackage{amsmath,amssymb,amsfonts,amsthm}
\usepackage{algorithmic}
\usepackage{graphicx}
\usepackage{textcomp}
\usepackage{xcolor}
\usepackage{txfonts}
\usepackage{listings}
\usepackage{enumitem}
\usepackage{mathtools}
\usepackage{gensymb}
\usepackage{comment}
\usepackage[breaklinks=true]{hyperref}
\usepackage{tkz-euclide} 
\usepackage{listings}
\usepackage{gvv}                                        
%\def\inputGnumericTable{}                                 
\usepackage[latin1]{inputenc}                                
\usepackage{color}                                            
\usepackage{array}                                            
\usepackage{longtable}                                       
\usepackage{calc}                                             
\usepackage{multirow}                                         
\usepackage{hhline}                                           
\usepackage{ifthen}                                           
\usepackage{lscape}
\usepackage{tabularx}
\usepackage{array}
\usepackage{float}
\usepackage{multicol}
\newcounter {sectioicolsn}
\newtheorem{theorem}{Theorem}[section]
\newtheorem{problem}{Problem}
\newtheorem{proposition}{Proposition}[section]
\newtheorem{lemma}{Lemma}[section]
\newtheorem{corollary}[theorem]{Corollary}
\newtheorem{example}{Example}[section]
\newtheorem{definition}[problem]{Definition}
\newcommand{\BEQA}{\begin{eqnarray}}
\newcommand{\EEQA}{\end{eqnarray}}
\newcommand{\define}{\stackrel{\triangle}{=}}
\theoremstyle{remark}
\newtheorem{rem}{Remark}
% Marks the beginning of the document
\begin{document}
\bibliographystyle{IEEEtran}
\vspace{3cm}
\title{31,August,2021 Shift-2 1-15}
\author{ee24btech11050 - Rahul Pothuri}
\maketitle
\newpage
\bigskip
\renewcommand{\thefigure}{\theenumi}
\renewcommand{\thetable}{\theenumi}

\begin{enumerate}[start=1]
\item %1
If $\alpha$ + $\beta$ + $\gamma$ = 2 $\pi$,Then the system of equations \\
$x+\brak{\cos{\gamma}}y+\brak{\cos{\beta}}z$=0 \\
$\brak{\cos{\gamma}}x+y+\brak{\cos{\alpha}}z$=0 \\
$\brak{\cos{\beta}}x+\brak{\cos{\alpha}}y+z$=0 \\
has :
\begin{enumerate}
\item no solution
\item infinitely many solutions
\item exactly two solutions
\item a unique solution
\end{enumerate}
\\


\item %2
let $\overrightarrow{a}$,$\overrightarrow{b}$,$\overrightarrow{c}$ be three vectors mutually perpendicular to each other and have same magnitude. If a vector $\overrightarrow{r}$ satisfies \\
$\overrightarrow{a} \times \{\brak{\overrightarrow{r}-\overrightarrow{b}}\times\overrightarrow{a}\}+\overrightarrow{b} \times \{\brak{\overrightarrow{r}-\overrightarrow{c}}\times\overrightarrow{b}\}+\overrightarrow{c} \times \{\brak{\overrightarrow{r}-\overrightarrow{a}}\times\overrightarrow{c}\} = \overrightarrow{o}$ \\
Then $\overrightarrow{r}$ is equal to : \\
\begin{enumerate}
\begin{multicols}{2}
\item
$\frac{1}{3} \left(\overrightarrow{a}+\overrightarrow{b}+\overrightarrow{c}\right)$
\item
$\frac{1}{3} \left(2\overrightarrow{a}+\overrightarrow{b}-\overrightarrow{c}\right)$
\end{multicols}
\end{enumerate}
\begin{enumerate}
\begin{multicols}{2}
\item
$\frac{1}{2} \left(\overrightarrow{a}+\overrightarrow{b}+\overrightarrow{c}\right)$
\item
$\frac{1}{2} \left(\overrightarrow{a}+\overrightarrow{b}+2\overrightarrow{c}\right)$
\end{multicols}
\end{enumerate}
\\

\item 
The domian of the function \\
f\brak{x}= $\sin ^- \left(\frac{3x^2+x-1}{\brak{x-1}^2}\right)+\cos ^- \left(\frac{x-1}{x+1}\right)$ is:
\begin{enumerate}
\begin{multicols}{2}
\item 
$\left[0,\frac{1}{4}\right]$
\item 
$\left[-2,0\right] \cup \left[\frac{1}{4},\frac{1}{2}\right]$
\end{multicols}
\end{enumerate}
\begin{enumerate}
\begin{multicols}{2}
\item 
$\left[\frac{1}{4},\frac{1}{2}\right] \cup \{0\} $
\item 
$\left[0,\frac{1}{2}\right]$
\end{multicols}
\end{enumerate}
\\


\item 
Let S= \{1,2,3,4,5,6\}. Then the probability that a randomly chosen onto function g from S to S satisfies $g\brak{3}=2g\brak{1}$ is :
\begin{enumerate}
\begin{multicols}{2}
\item $\frac{1}{10}$
\item $\frac{1}{15}$
\end{multicols}
\end{enumerate}
\begin{enumerate}
\begin{multicols}{2}
\item $\frac{1}{5}$
\item $\frac{1}{30}$
\end{multicols}
\end{enumerate}
\\

\item 
Let $f$ :$N \mapsto N$ be a  function such that $f\brak{m+n}=f\brak{m}+f\brak{n}$ for every $m,n \in N$. If $f\brak{6}=18$ then $f\brak{2}$.$f\brak{3}$ is equal to :
\begin{enumerate}
\begin{multicols}{2}
\item 6
\item 54
\end{multicols}
\end{enumerate}
\begin{enumerate}
\begin{multicols}{2}
\item 18
\item 36
\end{multicols}
\end{enumerate}
\\

\item 
The distance of the point \brak{-1,2,-2} from the line of intersection of the planes 2x+3y+2z=0 and x-2y+z=0 is:
\begin{enumerate}
\begin{multicols}{2}
\item $\frac{1}{\sqrt{2}}$
\item $\frac{5}{2}$
\end{multicols}
\end{enumerate}
\begin{enumerate}
\begin{multicols}{2}
\item $\frac{\sqrt{42}}{2}$
\item $\frac{\sqrt{34}}{2}$
\end{multicols}
\end{enumerate}
\\

\item 
Negation of the statement $\left(p \vee r \right) \implies \left(q \vee r \right)$ is :
\begin{enumerate}
\begin{multicols}{2}
\item $p \land \sim q \land \sim r$
\item $\sim p \land q \land r \sim$
\end{multicols}
\end{enumerate}
\begin{enumerate}
\begin{multicols}{2}
\item $ \sim p \land q \land r$
\item $ p \land q \land r $
\end{multicols}
\end{enumerate}
\\

\item 
If $\alpha= \lim_{x \to \frac{\pi}{4}} \frac{\tan^3{x}-\tan{x}}{\cos{\left(x+\frac{\pi}{4}\right)} }$ and $\beta = \lim_{x \to 0} \brak{\cos{x}}^{\cot{x}}$ are the roots of the equation $ax^2+bx-4=0$ , then the ordered pair \brak{a,b} is :
\begin{enumerate}
\begin{multicols}{2}
\item $\brak{1,-3}$
\item $\brak{-1,3}$
\end{multicols}
\end{enumerate}
\begin{enumerate}
\begin{multicols}{2}
\item $\brak{-1,-3}$
\item $\brak{1,3}$
\end{multicols}
\end{enumerate}
\\

\item 
The locus of the midpoints of the line segments joining \brak{-3,-5} and the points on the ellipse $\frac{x^2}{4}+\frac{y^2}{4}=1$ is :
\begin{enumerate}
\item $9x^2 + 4y^2 + 18x + 8y + 145=0$
\item $36x^2 + 16y^2 +90x +56y +145=0$
\item $36x^2 + 16y^2 +108x +80y +145=0$
\item $36x^2 + 16y^2 +72x +32y +145=0$
\end{enumerate}
\\

\item 
If $\frac{dy}{dx}= \frac{2^xy+2^y.2^x}{2^x+2^{x+y}\log_e 2}$,$y\brak{0} = 0$ , then for $y=1$ , the value of $x$ lies in the interval 
\begin{enumerate}
\begin{multicols}{2}
\item $\left(1,2\right)$
\item $\left(\frac{1}{2},1 \right]$
\end{multicols}
\end{enumerate}
\begin{enumerate}
\begin{multicols}{2}
\item $\left(2,3\right)$
\item $\left(0,\frac{1}{2}\right]$
\end{multicols}
\end{enumerate}

\newpage

\item 
An angle of intersection of the curves $\frac{x^2}{a^2}+\frac{y^2}{b^2}=1$ and $x^2+y^2=ab$ , $a > b$ is :
\begin{enumerate}
\begin{multicols}{2}
\item $\tan^-{\left(\frac{a+b}{\sqrt{ab}}\right)}$
\item $\tan^-{\left(\frac{a-b}{2\sqrt{ab}}\right)}$
\end{multicols}
\end{enumerate}
\begin{enumerate}
\begin{multicols}{2}
\item $\tan^-{\left(\frac{a-b}{\sqrt{ab}}\right)}$
\item $\tan^-{\left(2\sqrt{ab}\right)}$
\end{multicols}
\end{enumerate}
\\


\item 
If $y\frac{dy}{dx}=x\left[\frac{y^2}{x^2}+\frac{\phi\left(\frac{y^2}{x^2}\right)}{\phi'\left(\frac{y^2}{x^2}\right)}\right]$ , $x>0$, $\phi >0$, and $y\brak{1}=-1$, then $\phi\left(\frac{y^2}{4}\right)$ is equal to :
\begin{enumerate}
\begin{multicols}{2}
\item $4\phi \brak{2}$
\item $4\phi \brak{1}$
\end{multicols}
\end{enumerate}
\begin{enumerate}
\begin{multicols}{2}
\item $2\phi \brak{1}$
\item $\phi \brak{1}$
\end{multicols}
\end{enumerate}
\\

\item 
The sum of the roots of the equation \\ 
$x+1-2\log_4 \brak{3+2^x}+2\log_4 \brak{10-2^{-x}}=0$,is:
\begin{enumerate}
\begin{multicols}{2}
\item $log_2 14$
\item $log_2 11$
\end{multicols}
\end{enumerate}
\begin{enumerate}
\begin{multicols}{2}
\item $log_2 12$
\item $log_2 13$
\end{multicols}
\end{enumerate}
\\

\item 
If z is a complex number such that $\frac{z-i}{z-1}$ is purely imaginary the the minimum value of $\left| z-\brak{3+3i}\right|$ is :
\begin{enumerate}
\begin{multicols}{2}
\item $2\sqrt{2}-1$
\item $3\sqrt{2}$
\end{multicols}
\end{enumerate}
\begin{enumerate}
\begin{multicols}{2}
\item $6\sqrt{2}$
\item $2\sqrt{2}$
\end{multicols}
\end{enumerate}
\\

\item 
Let $a_1,a_2,a_3,.....$ be an AP. If $\frac{a_1+a_2+...+a_10}{a_1+a_2+...+a_p}  = \frac{100}{p^2}$ , $p\neq 10$, then $\frac{a_11}{a_10}$ is equal to :
\begin{enumerate}
\begin{multicols}{2}
\item $\frac{19}{21}$
\item $\frac{100}{121}$
\end{multicols}
\end{enumerate}
\begin{enumerate}
\begin{multicols}{2}
\item  $\frac{21}{19}$
\item $\frac{121}{100}$
\end{multicols}
\end{enumerate}
\\


\end{enumerate}
\end{document}