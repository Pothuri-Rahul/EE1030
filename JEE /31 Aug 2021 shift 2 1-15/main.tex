\let\negmedspace\undefined
\let\negthickspace\undefined
\documentclass[journal]{IEEEtran}
\usepackage[a5paper, margin=10mm, onecolumn]{geometry}
%\usepackage{lmodern} % Ensure lmodern is loaded for pdflatex
\usepackage{tfrupee} % Include tfrupee package
\setlength{\headheight}{1cm} % Set the height of the header box
\setlength{\headsep}{0mm}     % Set the distance between the header box and the top of the text
\usepackage{gvv-book}
\usepackage{gvv}
\usepackage{cite}
\usepackage{amsmath,amssymb,amsfonts,amsthm}
\usepackage{algorithmic}
\usepackage{graphicx}
\usepackage{textcomp}
\usepackage{xcolor}
\usepackage{txfonts}
\usepackage{listings}
\usepackage{enumitem}
\usepackage{mathtools}
\usepackage{gensymb}
\usepackage{comment}
\usepackage[breaklinks=true]{hyperref}
\usepackage{tkz-euclide} 
\usepackage{listings}
% \usepackage{gvv}                                        
\def\inputGnumericTable{}                                 
\usepackage[latin1]{inputenc}                                
\usepackage{color}                                            
\usepackage{array}                                            
\usepackage{longtable}                                       
\usepackage{calc}                                             
\usepackage{multirow}                                         
\usepackage{hhline}                                           
\usepackage{ifthen}                                           
\usepackage{lscape}
\usepackage[a5paper, margin=10mm, onecolumn]{geometry}
\usepackage{graphicx}
\begin{document}
\bibliographystyle{IEEEtran}
\vspace{3cm}
\title{31,August,2021 Shift-2 1-15}
\author{EE24BTECH11050 - Pothuri Rahul}
% \maketitle
% \newpage
% \bigskip
{\let\newpage\relax\maketitle}
\renewcommand{\thefigure}{\theenumi}
\renewcommand{\thetable}{\theenumi}
\setlength{\intextsep}{10pt} % Space between text and floats
\numberwithin{equation}{enumi}
\numberwithin{figure}{enumi}
\renewcommand{\thetable}{\theenumi}
\begin{enumerate}[start=1]
\item %1
If $\alpha$ + $\beta$ + $\gamma$ = 2$\pi$,Then the system of equations \\
$x+\brak{\cos{\gamma}}y+\brak{\cos{\beta}}z$=0 \\
$\brak{\cos{\gamma}}x+y+\brak{\cos{\alpha}}z$=0 \\
$\brak{\cos{\beta}}x+\brak{\cos{\alpha}}y+z$=0 \\
has :
\begin{enumerate}
\item no solution
\item infinitely many solutions
\item exactly two solutions
\item a unique solution
\end{enumerate}
\item %2
let $\overrightarrow{a}$,$\overrightarrow{b}$,$\overrightarrow{c}$ be three vectors mutually perpendicular to each other and have same magnitude. If a vector $\overrightarrow{r}$ satisfies \\
$\overrightarrow{a} \times \cbrak{\brak{\overrightarrow{r}-\overrightarrow{b}}\times\overrightarrow{a}}+\overrightarrow{b} \times \cbrak{\brak{\overrightarrow{r}-\overrightarrow{c}}\times\overrightarrow{b}}+\overrightarrow{c} \times \cbrak{\brak{\overrightarrow{r}-\overrightarrow{a}}\times\overrightarrow{c}} = \overrightarrow{o}$ \\
Then $\overrightarrow{r}$ is equal to : \\
\begin{enumerate}
\begin{multicols}{2}
\item
$\frac{1}{3} \brak{\overrightarrow{a}+\overrightarrow{b}+\overrightarrow{c}}$
\item
$\frac{1}{3} \brak{2\overrightarrow{a}+\overrightarrow{b}-\overrightarrow{c}}$
\end{multicols}
\begin{multicols}{2}
\item
$\frac{1}{2} \brak{\overrightarrow{a}+\overrightarrow{b}+\overrightarrow{c}}$
\item
$\frac{1}{2} \brak{\overrightarrow{a}+\overrightarrow{b}+2\overrightarrow{c}}$
\end{multicols}
\end{enumerate}
\item %3
The domain of the function \\
$f\brak{x}$= $\sin ^{-1} \brak{\frac{3x^2+x-1}{\brak{x-1}^2}}+\cos ^{-1} \brak{\frac{x-1}{x+1}}$ is:
\begin{enumerate}
\begin{multicols}{2}
\item 
$\sbrak{0,\frac{1}{4}}$
\item 
$\sbrak{-2,0} \cup \sbrak{\frac{1}{4},\frac{1}{2}}$
\end{multicols}
\begin{multicols}{2}
\item 
$\sbrak{\frac{1}{4},\frac{1}{2}} \cup \cbrak{0} $
\item 
$\sbrak{0,\frac{1}{2}}$
\end{multicols}
\end{enumerate}
\item %4
Let $S= \cbrak{1,2,3,4,5,6}$. Then the probability that a randomly chosen onto function $g$ from $S$ to $S$ satisfies $g\brak{3}=2g\brak{1}$ is :
\begin{enumerate}
\begin{multicols}{2}
\item $\frac{1}{10}$
\item $\frac{1}{15}$
\end{multicols}
\begin{multicols}{2}
\item $\frac{1}{5}$
\item $\frac{1}{30}$
\end{multicols}
\end{enumerate}
\item %5
Let $f$ :$ \mathbb{N} \mapsto \mathbb{N}$ be a  function such that $f\brak{m+n}=f\brak{m}+f\brak{n}$ for every $m,n \in \mathbb{N}$. If $f\brak{6}=18$ then $f\brak{2}$.$f\brak{3}$ is equal to :
\begin{enumerate}
\begin{multicols}{2}
\item $6$
\item $54$
\end{multicols}
\begin{multicols}{2}
\item $18$
\item $36$
\end{multicols}
\end{enumerate}
\item %6
The distance of the point $\brak{-1,2,-2}$ from the line of intersection of the planes $2x+3y+2z=0$ and $x-2y+z=0$ is:
\begin{enumerate}
\begin{multicols}{2}
\item $\frac{1}{\sqrt{2}}$
\item $\frac{5}{2}$
\end{multicols}
\begin{multicols}{2}
\item $\frac{\sqrt{42}}{2}$
\item $\frac{\sqrt{34}}{2}$
\end{multicols}
\end{enumerate}
\item %7
Negation of the statement $\brak{p \vee r } \implies \brak{q \vee r }$ is :
\begin{enumerate}
\begin{multicols}{2}
\item $p \land \sim q \land \sim r$
\item $\sim p \land q \land r \sim$
\end{multicols}
\begin{multicols}{2}
\item $ \sim p \land q \land r$
\item $ p \land q \land r $
\end{multicols}
\end{enumerate}
\item %8
If $\alpha = \lim\limits_{x \to \frac{\pi}{4}} \frac{\tan^3{x}-\tan{x}}{\cos{\brak{x+\frac{\pi}{4}}} }$ and $\beta = \lim\limits_{x \to 0} \brak{\cos{x}}^{\cot{x}}$ are the roots of the equation $ax^2+bx-4=0$ , then the ordered pair $\brak{a,b}$ is :
\begin{enumerate}
\begin{multicols}{2}
\item $\brak{1,-3}$
\item $\brak{-1,3}$
\end{multicols}
\begin{multicols}{2}
\item $\brak{-1,-3}$
\item $\brak{1,3}$
\end{multicols}
\end{enumerate}
\item %9
The locus of the midpoints of the line segments joining $\brak{-3,-5}$ and the points on the ellipse $\frac{x^2}{4}+\frac{y^2}{4}=1$ is :
\begin{enumerate}
\item $9x^2 + 4y^2 + 18x + 8y + 145=0$
\item $36x^2 + 16y^2 +90x +56y +145=0$
\item $36x^2 + 16y^2 +108x +80y +145=0$
\item $36x^2 + 16y^2 +72x +32y +145=0$
\end{enumerate}
\item %10
If $\frac{dy}{dx}= \frac{2^xy+2^y.2^x}{2^x+2^{x+y}\log_e 2}$,$y\brak{0} = 0$ , then for $y=1$ , the value of $x$ lies in the interval 
\begin{enumerate}
\begin{multicols}{2}
\item $\brak{1,2}$
\item $(\frac{1}{2},1]$
\end{multicols}
\begin{multicols}{2}
\item $\brak{2,3}$
\item $(0,\frac{1}{2}]$
\end{multicols}
\end{enumerate}
\item %11
An angle of intersection of the curves $\frac{x^2}{a^2}+\frac{y^2}{b^2}=1$ and $x^2+y^2=ab$ , $a > b$ is :
\begin{enumerate}
\begin{multicols}{2}
\item $\tan^{-1}{\brak{\frac{a+b}{\sqrt{ab}}}}$
\item $\tan^{-1}{\brak{\frac{a-b}{2\sqrt{ab}}}}$
\end{multicols}
\begin{multicols}{2}
\item $\tan^{-1}{\brak{\frac{a-b}{\sqrt{ab}}}}$
\item $\tan^{-1}{\brak{2\sqrt{ab}}}$
\end{multicols}
\end{enumerate}
\item %12
If $y\frac{dy}{dx}=x\sbrak{\frac{y^2}{x^2}+\frac{\phi\brak{\frac{y^2}{x^2}}}{\phi'\brak{\frac{y^2}{x^2}}}}$ , $x>0$, $\phi >0$, and $y\brak{1}=-1$, then $\phi\brak{\frac{y^2}{4}}$ is equal to :
\begin{enumerate}
\begin{multicols}{2}
\item $4\phi \brak{2}$
\item $4\phi \brak{1}$
\end{multicols}
\begin{multicols}{2}
\item $2\phi \brak{1}$
\item $\phi \brak{1}$
\end{multicols}
\end{enumerate}
\item %13
The sum of the roots of the equation \\ 
$x+1-2\log_2 \brak{3+2^x}+2\log_4 \brak{10-2^{-x}}=0$, is:
\begin{enumerate}
\begin{multicols}{2}
\item $\log_2 14$
\item $\log_2 11$
\end{multicols}
\begin{multicols}{2}
\item $\log_2 12$
\item $\log_2 13$
\end{multicols}
\end{enumerate}
\item %14
If $z$ is a complex number such that $\frac{z-i}{z-1}$ is purely imaginary the the minimum value of $\abs{ z-\brak{3+3i}}$ is :
\begin{enumerate}
\begin{multicols}{2}
\item $2\sqrt{2}-1$
\item $3\sqrt{2}$
\end{multicols}
\begin{multicols}{2}
\item $6\sqrt{2}$
\item $2\sqrt{2}$
\end{multicols}
\end{enumerate}
\item %15
Let $a_{1},a_{2},a_{3},.....$ be an AP. If $\frac{a_{1}+a_{2}+...+a_{10}}{a_{1}+a_{2}+...+a_{p}}  = \frac{100}{p^2}$ , $p\neq 10$, then $\frac{a_{11}}{a_{10}}$ is equal to :
\begin{enumerate}
\begin{multicols}{2}
\item $\frac{19}{21}$
\item $\frac{100}{121}$
\end{multicols}
\begin{multicols}{2}
\item  $\frac{21}{19}$
\item $\frac{121}{100}$
\end{multicols}
\end{enumerate}
\end{enumerate}
\end{document}