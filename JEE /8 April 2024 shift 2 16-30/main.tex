\let\negmedspace\undefined
\let\negthickspace\undefined
\documentclass[journal]{IEEEtran}
\usepackage[a5paper, margin=10mm, onecolumn]{geometry}
%\usepackage{lmodern} % Ensure lmodern is loaded for pdflatex
\usepackage{tfrupee} % Include tfrupee package
\setlength{\headheight}{1cm} % Set the height of the header box
\setlength{\headsep}{0mm}     % Set the distance between the header box and the top of the text
\usepackage{gvv-book}
\usepackage{gvv}
\usepackage{cite}
\usepackage{amsmath,amssymb,amsfonts,amsthm}
\usepackage{algorithmic}
\usepackage{graphicx}
\usepackage{textcomp}
\usepackage{xcolor}
\usepackage{txfonts}
\usepackage{listings}
\usepackage{enumitem}
\usepackage{mathtools}
\usepackage{gensymb}
\usepackage{comment}
\usepackage[breaklinks=true]{hyperref}
\usepackage{tkz-euclide} 
\usepackage{listings}
% \usepackage{gvv}                                        
\def\inputGnumericTable{}                                 
\usepackage[latin1]{inputenc}                                
\usepackage{color}                                            
\usepackage{array}                                            
\usepackage{longtable}                                       
\usepackage{calc}                                             
\usepackage{multirow}                                         
\usepackage{hhline}                                           
\usepackage{ifthen}                                           
\usepackage{lscape}
\usepackage[a5paper, margin=10mm, onecolumn]{geometry}
\usepackage{graphicx}
\begin{document}
\bibliographystyle{IEEEtran}
\vspace{3cm}
\title{8,April,2024 Shift-2 16-30}
\author{EE24BTECH11050 - Pothuri Rahul}
% \maketitle
% \newpage
% \bigskip
{\let\newpage\relax\maketitle}
\renewcommand{\thefigure}{\theenumi}
\renewcommand{\thetable}{\theenumi}
\setlength{\intextsep}{10pt} % Space between text and floats
\numberwithin{equation}{enumi}
\numberwithin{figure}{enumi}
\renewcommand{\thetable}{\theenumi}
\begin{enumerate}[start=16]
\item %16
If the line segment joining the points $\brak{5,2}$ and $\brak{2,a}$ subtends an angle $\frac{\pi}{4}$ at the origin, then the absolute value of the product of all possible values of $a$ is :
\begin{enumerate}
\begin{multicols}{2}
\item $4$
\item $2$
\end{multicols}
\begin{multicols}{2}
\item $6$
\item $8$
\end{multicols}
\end{enumerate}
\item %17
If the shortest distance between the lines $\frac{x - \lambda}{2} = \frac{y - 4}{3} = \frac{z - 3}{4}$ and $\frac{x - 2}{4} = \frac{y - 4}{6} = \frac{z - 7}{8}$ is $\frac{13}{\sqrt{29}}$, then the value of $\lambda$ is :
\begin{enumerate}
\begin{multicols}{2}
\item $-1$
\item $1$
\end{multicols}
\begin{multicols}{2}
\item $\frac{13}{25}$
\item $ - \frac{13}{25}$
\end{multicols}
\end{enumerate}
\item %18
The number of ways five alphabets can be chosen from the alphabets of the word MATHEMATICS, where the chosen alphabets are not necessarily distinct, is equal to :
\begin{enumerate}
\begin{multicols}{2}
\item $175$
\item $179$
\end{multicols}
\begin{multicols}{2}
\item $181$
\item $177$
\end{multicols}
\end{enumerate}
\item %19
if $\alpha \neq a, \beta \neq b, \gamma \neq c$ and  
$\abs{\begin{array}{ccc}
\alpha & b & c \\ a & \beta & c \\ a & b & \gamma \end{array}} =0 $, then $\frac{a}{\alpha - a}+\frac{b}{\beta - b}+\frac{\gamma}{\gamma - c}$ is equal to :
\begin{enumerate}
\begin{multicols}{2}
\item $2$
\item $3$
\end{multicols}
\begin{multicols}{2}
\item $0$
\item $1$
\end{multicols}
\end{enumerate}
\item %20
For $a, b >0$,  let $f\brak{x}= 
\begin{cases}
\frac{\tan{\brak{\brak{a+1}x}}+b\tan{x}}{x} &, x < 0; \\
3 &,  x = 0; \\
\frac{\sqrt{ax+b^2x^2} - \sqrt{ax}}{b\sqrt{a}x\sqrt{x}} &,  x>0;
\end{cases} $ 
be a continuous function at $x=0$. Then $\frac{b}{a}$ is equal to :
\begin{enumerate}
\begin{multicols}{2}
\item $5$
\item $4$
\end{multicols}
\begin{multicols}{2}
\item $6$
\item $8$
\end{multicols}
\end{enumerate}
\item %21
Let a ray of light passing through the point $\brak{3,10}$ reflects on the line $2x+y=6$ and then reflected ray passes through the point $\brak{7,2}$. If the equation of the incident ray is $ax+by+1=0$, then $a^2+b^2+3ab$ is equal to \underline{\hspace{1cm}}.
\item %22
Let $\alpha \abs{x} = \abs{y} e^{xy - \beta}, \alpha , \beta \in \mathbb{N} $ be the solution of the differential equation $xdy-ydx+xy\brak{xdy+ydx} = 0$, $y\brak{1} = 2 $. Then $\alpha + \beta $ is equal to \underline{\hspace{1cm}}.
\item %23
Let $a,b,c \in \mathbb{N}$ and $a<b<c$. Let the mean, the mean deviation about the mean and the variance of the 5 observations $9,25,a,b,c$ be 18 , 4 and $\frac{136}{5}$, respectively. Then $2a+b-c$ is equal to \underline{\hspace{1cm}}.
\item %24
Let $S$ be the focus of the hyperbola $\frac{x^2}{3} - \frac{y^2}{5} = 1$, on the positive x-axis. Let $C$ be the circle its centre at $A\brak{\sqrt{6},\sqrt{5}}$ and passing through the points $S$. If $O$ is the origin and $SAB$ is a diameter of $C$, then the square of the area of the triangle $OSB$ is equal to \underline{\hspace{1cm}}. 
\item %25
Let $A$ be the region enclosed by the parabola $y^2 = 2x$ and the line $x=24$. Then the maximum area of the rectangle inscribed in the region A is: \underline{\hspace{1cm}}.
\item %26
An arithmetic progression is written in the following way 
$
\begin{array}{ccccccc}
& & & 2\\
& & 5 & & 8 & & \\
& 11 &  & 14 & & 17 & \\
20 & & 23 & & 26 & & 29  \\
\_ \_ \_ \_ & \_ \_ \_ \_ & \_ \_ \_ \_ & \_ \_ \_ \_ & \_ \_ \_ \_ & \_ \_ \_ \_ & \_ \_ \_ \_ \\
\_ \_ \_ \_ & \_ \_ \_ \_ & \_ \_ \_ \_ & \_ \_ \_ \_ & \_ \_ \_ \_ & \_ \_ \_ \_ & \_ \_ \_ \_ 
\end{array}
$\\
\\
The sum of all the terms in 10th row is \underline{\hspace{1cm}}.
\item %27
The number of distinct real roots of the equation $\abs{x + 1} \abs{x + 3} - 4\abs{x + 2} + 5 = 0$ is \underline{\hspace{1cm}}.
\item %28
If $\alpha = \lim\limits_{x \to 0^+} \brak{\frac{e^{\sqrt{\tan{x}}} - e^{\sqrt{x}}}{\sqrt{\tan{x}}-\sqrt{x}}}$ and $\beta = \lim\limits_{x \to 0}\brak{1+\sin{x}}^{\frac{1}{2}\cot{x}}$ are the roots of the quadratic equation $ax^2+bx-\sqrt{e} = 0$, then $12 \log_e \brak{a+b}$ is equal to \underline{\hspace{1cm}}. 
\item %29 
If $  \int \frac{1}{\sqrt[5]{\brak{x-1}^4\brak{x+3}^6}} dx = A\brak{\frac{\alpha x -1}{\beta x + 3}}^B + C$, where C is constant of integration, then the value of $\alpha + \beta +20AB$ is \underline{\hspace{1cm}}.
\item %30
let $P\brak{\alpha , \beta , \gamma }$ be the image of the point $Q\brak{1 , 6, 4}$ in the line $\frac{x}{1} = \frac{y-1}{2} = \frac{z-2}{3}$. Then $2\alpha +\beta +\gamma$ is equal to \underline{\hspace{1cm}}.
\end{enumerate}
\end{document}