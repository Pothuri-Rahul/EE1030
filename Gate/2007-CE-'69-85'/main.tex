\let\negmedspace\undefined
\let\negthickspace\undefined
\documentclass[journal]{IEEEtran}
\usepackage[a5paper, margin=10mm, onecolumn]{geometry}
%\usepackage{lmodern} % Ensure lmodern is loaded for pdflatex
\usepackage{tfrupee} % Include tfrupee package
\setlength{\headheight}{1cm} % Set the height of the header box
\setlength{\headsep}{0mm}     % Set the distance between the header box and the top of the text
\usepackage{gvv-book}
\usepackage{gvv}
\usepackage{cite}
\usepackage{amsmath,amssymb,amsfonts,amsthm}
\usepackage{algorithmic}
\usepackage{graphicx}
\usepackage{textcomp}
\usepackage{xcolor}
\usepackage{tikz}
\usepackage{txfonts}
\usepackage{listings}
\usepackage{enumitem}
\usepackage{mathtools}
\usepackage{gensymb}
\usepackage{comment}
\usepackage[breaklinks=true]{hyperref}
\usepackage{tkz-euclide} 
\usepackage{listings}
% \usepackage{gvv}                                        
\def\inputGnumericTable{}                                 
\usepackage[latin1]{inputenc}                                
\usepackage{color}                                            
\usepackage{array}                                            
\usepackage{longtable}                                       
\usepackage{calc}                                             
\usepackage{multirow}                                         
\usepackage{hhline}                                           
\usepackage{ifthen}                                           
\usepackage{lscape}
\usepackage[a5paper, margin=10mm, onecolumn]{geometry}
\usepackage{graphicx}
\begin{document}
\bibliographystyle{IEEEtran}
\vspace{3cm}
\title{2007-CE-69-85}
\author{EE24BTECH11050 - Pothuri Rahul}
% \maketitle
% \newpage
% \bigskip
{\let\newpage\relax\maketitle}
\renewcommand{\thefigure}{\theenumi}
\renewcommand{\thetable}{\theenumi}
\setlength{\intextsep}{10pt} % Space between text and floats
\numberwithin{equation}{enumi}
\numberwithin{figure}{enumi}
\renewcommand{\thetable}{\theenumi}

\begin{enumerate}[start=69]


\item %69
The magnetic bearing of a line AB is S $45\degree$ E and the declination is $5\degree$ West. The true bearing of the line AB is 
\begin{enumerate}
\begin{multicols}{4}
\item S $45\degree$ E
\item S $40\degree$ E
\item S $50\degree$ E
\item S $50\degree$ W
\end{multicols}
\end{enumerate}
\end{enumerate}

\begin{center}
    \textbf{COMMON DATA QUESTIONS}
\end{center}

%70&71
\textbf{Common data for questions 70 \& 71 }\\
Water is flowing through the permeability apparatus as shown in figure. The coefficient of permeability of the soil is k m/s and the porosity of the soil sample is 0.50.


\begin{figure}[!ht]
\centering
\resizebox{1\textwidth}{!}{%
\begin{circuitikz}
\tikzstyle{every node}=[font=\large]
\draw [ line width=1pt](6,14.5) to[short] (9.25,14.5);
\draw [ line width=1pt](6,14.5) to[short] (6,11.75);
\draw [ line width=1pt](9.25,14.5) to[short] (9.25,11.25);
\draw [ line width=1pt](6,11.75) to[short] (6,11.25);
\draw [ line width=1pt](6,11.25) to[short] (9.25,11.25);
\draw [ line width=1pt](6,13.25) to[short] (9.25,13.25);
\draw [ line width=1pt](7.25,11.25) to[short] (7.25,10.25);
\draw [ line width=1pt](8,11.25) to[short] (8,10.5);
\draw [ line width=1pt](8,10.5) to[short] (11.75,10.5);
\draw [ line width=1pt](7.25,10.25) to[short] (7.25,9.75);
\draw [ line width=1pt](7.25,9.75) to[short] (12.75,9.75);
\draw [ line width=1pt](11.75,10.5) to[short] (11.75,11.25);
\draw [ line width=1pt](12.75,11.25) to[short] (12.75,9.75);
\draw [ line width=1pt](10.75,11.25) to[short] (14,11.25);
\draw [ line width=1pt](10.75,11.25) to[short] (10.75,12.25);
\draw [ line width=1pt](10.75,12.25) to[short] (11.75,12.25);
\draw [ line width=1pt](10.75,12.25) to[short] (14,12.25);
\draw [ line width=1pt](14,12.25) to[short] (14,11.25);
\draw [line width=0.7pt, dashed] (6,14.25) -- (9.25,14.25);
\draw [line width=0.7pt, dashed] (6,14) -- (9.25,14);
\draw [line width=0.7pt, dashed] (6,13.75) -- (9.25,13.75);
\draw [line width=0.7pt, dashed] (6,13.5) -- (9.25,13.5);
\draw [line width=0.7pt, dashed] (7.25,11) -- (8,11);
\draw [line width=0.7pt, dashed] (7.25,10.75) -- (8,10.75);
\draw [line width=0.7pt, dashed] (7.25,10.5) -- (8,10.5);
\draw [line width=0.7pt, dashed] (7.25,10.25) -- (12.75,10.25);
\draw [line width=0.7pt, dashed] (7.25,10) -- (12.75,10);
\draw [line width=0.7pt, dashed] (7.25,9.75) -- (12.75,9.75);
\draw [line width=0.7pt, dashed] (11.75,11) -- (12.75,11);
\draw [line width=0.7pt, dashed] (11.75,11) -- (12.75,11);
\draw [line width=0.7pt, dashed] (11.75,10.75) -- (12.75,10.75);
\draw [line width=0.7pt, dashed] (11.75,10.5) -- (12.5,10.5);
\draw [line width=0.7pt, dashed] (10.75,12) -- (14,12);
\draw [line width=0.7pt, dashed] (10.75,11.75) -- (14,11.75);
\draw [line width=0.7pt, dashed] (10.75,11.5) -- (14,11.5);
\draw [line width=0.7pt, short] (6.25,13.25) -- (6,13);
\draw [line width=0.7pt, short] (6.25,13.25) -- (6,13);
\draw [line width=0.7pt, short] (6.5,13.25) -- (6,12.75);
\draw [line width=0.7pt, short] (6.75,13.25) -- (6,12.5);
\draw [line width=0.7pt, short] (7,13.25) -- (6,12.25);
\draw [line width=0.7pt, short] (7.25,13.25) -- (6,12);
\draw [line width=0.7pt, short] (7.5,13.25) -- (6,11.75);
\draw [line width=0.7pt, short] (7.75,13.25) -- (6,11.5);
\draw [line width=0.7pt, short] (8,13.25) -- (6,11.25);
\draw [line width=0.7pt, short] (8.25,13.25) -- (6.25,11.25);
\draw [line width=0.7pt, short] (8.5,13.25) -- (6.5,11.25);
\draw [line width=0.7pt, short] (8.75,13.25) -- (6.75,11.25);
\draw [line width=0.7pt, short] (9,13.25) -- (7,11.25);
\draw [line width=0.7pt, short] (9,13.25) -- (7,11.25);
\draw [line width=0.7pt, short] (9.25,13.25) -- (7.25,11.25);
\draw [line width=0.7pt, short] (9.25,13) -- (7.5,11.25);
\draw [line width=0.7pt, short] (9.25,12.75) -- (7.75,11.25);
\draw [line width=0.7pt, short] (9.25,12.5) -- (8,11.25);
\draw [line width=0.7pt, short] (9.25,12.25) -- (8.25,11.25);
\draw [line width=0.7pt, short] (9.25,12) -- (8.5,11.25);
\draw [line width=0.7pt, short] (9.25,11.75) -- (8.75,11.25);
\draw [line width=0.7pt, short] (9.25,11.5) -- (9,11.25);
\draw [line width=0.7pt, short] (12,12.5) -- (12.25,12.25);
\draw [line width=0.7pt, short] (12.25,12.25) -- (12.5,12.5);
\draw [line width=0.7pt, short] (12,12.5) -- (12.5,12.5);
\draw [line width=0.7pt, short] (7.25,14.75) -- (7.5,14.5);
\draw [line width=0.7pt, short] (7.75,14.75) -- (7.5,14.5);
\draw [line width=0.7pt, short] (7.25,14.75) -- (7.75,14.75);
\draw [line width=0.6pt, dashed] (12.25,12.25) -- (7.5,12.25);
\draw [line width=0.6pt, short] (9.5,14.5) -- (10,14.5);
\draw [line width=0.6pt, short] (9.5,11.25) -- (10,11.25);
\draw [line width=0.6pt, <->, >=Stealth] (9.75,14.5) -- (9.75,13.25);
\draw [line width=0.6pt, <->, >=Stealth] (9.75,13.25) -- (9.75,12.25);
\draw [line width=0.6pt, <->, >=Stealth] (9.75,12.25) -- (9.75,11.25);
\draw [line width=0.6pt, short] (9.5,13.25) -- (10,13.25);
\node [font=\normalsize] at (10.25,13.75) {0.4m};
\node [font=\normalsize] at (10.25,12.75) {0.8m};
\node [font=\normalsize] at (10.25,11.75) {0.4m};
\node [font=\large] at (7.5,12.25) {\textbf{R}};
\node [font=\large] at (7.5,12.75) {\textbf{soil}};
\end{circuitikz}
}%

\label{fig:my_label}
\end{figure}




\begin {enumerate}[start=70]
\item %70
The total head,elevation head and pressure head in metres of water at the point R shown in the figure are
\begin{enumerate}
\begin{multicols}{4}
\item 0.8,0.4,0.4
\item 1.2,0.4,0.8
\item 0.4,0,0.4
\item 1.6,0.4,1.2
\end{multicols}
\end{enumerate}

\item %71
What are the discharge velocity and seepage velocity through the soil sample?
\begin{enumerate}
\begin{multicols}{4}
\item $k,2k$
\item $\frac{2}{3}k,\frac{4}{3}k$007-CE-'69-85
\item $2k,k$
\item $\frac{4}{3}k, \frac{2}{3}k$
\end{multicols}
\end{enumerate}


%72&73
\textbf{Common data for questions 72 \& 73} \\
Ordinates of a 1-hour unit hydrograph at 1 hour intervals, starting from t=0 are 0,2,6,4,2,1 and 0 $m^3$/s.

\item %72
Catchment area represented by this unit hydrograph is 
\begin{enumerate}
\begin{multicols}{4}
\item 1.0 $km^2$
\item 2.0 $km^2$
\item 3.2 $km^2$
\item 5.4 $km^2$
\end{multicols}
\end{enumerate}

\item %73
Ordinate of a 3-hour unit hydrograph for the catchment at t=3 hours is 
\begin{enumerate}
\begin{multicols}{4}
\item 2.0 $m^3/s$
\item 3.0 $m^3/s$
\item 4.0 $m^3/s$
\item 5.0 $m^3/s$
\end{multicols}
\end{enumerate}
\end{enumerate}

%74&75
\textbf{Common data for questions 74 \& 75} \\
A completely mixed activated sludge process is used to treat a wastewater flow of 1 million litres per day \brak{1 MLD} having a $BOD_5$ of 200 mg/L. The biomass concentration in the aeration tank is 2000 mg/L and the concentration of the next biomass leaving the system is 50 mg/L. The aeration tank has a volume of 200 $m^3$.
\begin{enumerate}[start=74]
\item %74
What is the hydraulic retention time of the wastewater in the aeration tank?
\begin{enumerate}
\begin{multicols}{4}
\item 0.2 h
\item 4.8 h
\item 10 h
\item 24 h
\end{multicols}
\end{enumerate}

\item %75
What is the average tie for which the biomass stays in the system?
\begin{enumerate}
\begin{multicols}{4}
\item 5 h
\item 8 h
\item 2 days
\item 8 days
\end{multicols}
\end{enumerate}

\begin{center}
    \textbf{LINKED ANSWER QUESTIONS:Q.76 TO Q.85 carry two marks each}
\end{center} 
\textbf{Statement for Linked Answer Questions 76 and 77:}\\
A two span continuous beam having equal spans each of length $L$ is subjected  to a uniformly distributed load $w$ per unit length. The beam has constant flexural rigidity.
\item %76
The reaction at the middle support is 
\begin{enumerate}
\begin{multicols}{4}
\item $wL$
\item $\frac{5wL}{2}$
\item $\frac{5wL}{4}$
\item $\frac{5wL}{8}$
\end{multicols}
\end{enumerate}

\item %77
The bending moment at the middle support is 
\begin{enumerate}
\begin{multicols}{4}
\item $\frac{wl^2}{4}$
\item $\frac{wl^2}{8}$
\item $\frac{wl^2}{12}$
\item $\frac{wl^2}{16}$
\end{multicols}
\end{enumerate}


\textbf{Statement for Linked Answer Questions 78 and 79:}\\
A singly reinforced rectangulatr concrete beam has a width of 150 mm and an effective depth of 330 mm. The characteristic compressive strength of concrete is 20 MPa and the characteristic tensile strength of steel is 415 MPa. Adopt the stress block for concrete as given in IS 456-2000 and take limiting value of depth of neutral axis as 0.48 times the effective depth of the beam. 
\item %78
The limiting value of the moment of resistance of the beam in kN.m is\\
\begin{enumerate}
\begin{multicols}{4}
\item 0.14
\item 0.45
\item 45.08
\item 156.82
\end{multicols}
\end{enumerate}

\item %79
The limiting area of tension steel in $mm^2$ is 
\begin{enumerate}
\begin{multicols}{4}
\item 473.9
\item 412.3
\item 373.9
\item 312.3
\end{multicols}
\end{enumerate}

\textbf{Statement for Linked Answer Questions 80 and 81:}\\
The ground conditions at a site are as shown in the figure. The water table at the site which was initially at a depth of 5m below the ground level got permanently lowered to a depth of 15m below the ground level due to pumping of water over a few years. Assume the following data: \\
\begin{itemize}
    \item[i).] unit weight of water = 10 kN/$m^3$
    \item[ii).] unit weight of sand above water table=18kN/$m^3$
    \item[iii).] unit weight of sand and clay below the water table= 20kN/$m^3$
    \item[iv).] coefficient of volume compressibility=0.25 $m^2$/MN
\end{itemize}

\begin{figure}[!ht]
\centering
\resizebox{1\textwidth}{!}{%
\begin{circuitikz}
\tikzstyle{every node}=[font=\normalsize]
\draw [line width=1pt, short] (4,13.75) -- (16.5,13.75);
\draw [line width=1pt, short] (3.75,8.5) -- (16.75,8.5);
\draw [line width=1pt, short] (3.75,7.25) -- (16.75,7.25);
\draw [line width=0.6pt, ->, >=Stealth] (4,11.25) -- (4,13.25);
\draw [line width=0.6pt, ->, >=Stealth] (4,10.75) -- (4,9);
\draw [line width=0.6pt, dashed] (11.75,12.75) -- (16.5,12.75);
\draw [line width=0.6pt, dashed] (12,9.5) -- (16.5,9.5);
\draw [line width=0.6pt, <->, >=Stealth] (16.25,8.5) -- (16.25,7.25);
\draw [line width=0.6pt, <->, >=Stealth] (16,13.75) -- (16,12.75);
\draw [line width=0.6pt, short] (4.25,13.75) -- (4,13.5);
\draw [line width=0.6pt, short] (4.5,13.75) -- (4.25,13.5);
\draw [line width=0.6pt, short] (4.75,13.75) -- (4.5,13.5);
\draw [line width=0.6pt, short] (5,13.75) -- (4.75,13.5);
\draw [line width=0.6pt, short] (5.25,13.75) -- (5,13.5);
\draw [line width=0.6pt, short] (5.5,13.75) -- (5.25,13.5);
\draw [line width=0.6pt, short] (5.75,13.75) -- (5.5,13.5);
\draw [line width=0.6pt, short] (6,13.75) -- (5.75,13.5);
\draw [line width=0.6pt, short] (6.25,13.75) -- (6,13.5);
\draw [line width=0.6pt, short] (6.5,13.75) -- (6.25,13.5);
\draw [line width=0.6pt, short] (6.75,13.75) -- (6.5,13.5);
\draw [line width=0.6pt, short] (7,13.75) -- (6.75,13.5);
\draw [line width=0.6pt, short] (7.25,13.75) -- (7,13.5);
\draw [line width=0.6pt, short] (7.5,13.75) -- (7.25,13.5);
\draw [line width=0.6pt, short] (7.75,13.75) -- (7.5,13.5);
\draw [line width=0.6pt, short] (8,13.75) -- (7.75,13.5);
\draw [line width=0.6pt, short] (8.25,13.75) -- (8,13.5);
\draw [line width=0.6pt, short] (8.5,13.75) -- (8.25,13.5);
\draw [line width=0.6pt, short] (8.75,13.75) -- (8.5,13.5);
\draw [line width=0.6pt, short] (9,13.75) -- (8.75,13.5);
\draw [line width=0.6pt, short] (9.25,13.75) -- (9,13.5);
\draw [line width=0.6pt, short] (9.5,13.75) -- (9.25,13.5);
\draw [line width=0.6pt, short] (9.75,13.75) -- (9.5,13.5);
\draw [line width=0.6pt, short] (10,13.75) -- (9.75,13.5);
\draw [line width=0.6pt, short] (10.25,13.75) -- (10,13.5);
\draw [line width=0.6pt, short] (10.5,13.75) -- (10.25,13.5);
\draw [line width=0.6pt, short] (10.75,13.75) -- (10.5,13.5);
\draw [line width=0.6pt, short] (11,13.75) -- (10.75,13.5);
\draw [line width=0.6pt, short] (11.25,13.75) -- (11,13.5);
\draw [line width=0.6pt, short] (11.25,13.75) -- (11,13.5);
\draw [line width=0.6pt, short] (11.5,13.75) -- (11.25,13.5);
\draw [line width=0.6pt, short] (11.75,13.75) -- (11.5,13.5);
\draw [line width=0.6pt, short] (12,13.75) -- (11.75,13.5);
\draw [line width=0.6pt, short] (12.25,13.75) -- (12,13.5);
\draw [line width=0.6pt, short] (12.5,13.75) -- (12.25,13.5);
\draw [line width=0.6pt, short] (12.75,13.75) -- (12.5,13.5);
\draw [line width=0.6pt, short] (13,13.75) -- (12.75,13.5);
\draw [line width=0.6pt, short] (13.25,13.75) -- (13,13.5);
\draw [line width=0.6pt, short] (13.5,13.75) -- (13.25,13.5);
\draw [line width=0.6pt, short] (13.75,13.75) -- (13.5,13.5);
\draw [line width=0.6pt, short] (14,13.75) -- (13.75,13.5);
\draw [line width=0.6pt, short] (14.25,13.75) -- (14,13.5);
\draw [line width=0.6pt, short] (14.5,13.75) -- (14.25,13.5);
\draw [line width=0.6pt, short] (14.75,13.75) -- (14.5,13.5);
\draw [line width=0.6pt, short] (15,13.75) -- (14.75,13.5);
\draw [line width=0.6pt, ->, >=Stealth] (15,11.75) -- (15,13.75);
\draw [line width=0.6pt, ->, >=Stealth] (15,11) -- (15,9.5);
\node [font=\large] at (15,11.25) {\textbf{15m}};
\node [font=\large] at (16.5,13.25) {\textbf{5m }};
\node [font=\large] at (16.75,7.75) {\textbf{5m }};
\node [font=\large] at (4,11) {\textbf{20m }};
\node [font=\large] at (6.5,7.75) {\textbf{Clay soil layer}};
\node [font=\normalsize] at (7.25,9.5) {\textbf{lowered water table }};
\node [font=\normalsize] at (7.5,12.75) {\textbf{original water table }};
\draw [line width=0.6pt, short] (12.25,13) -- (12.5,12.75);
\draw [line width=0.6pt, short] (12.75,13) -- (12.5,12.75);
\draw [line width=0.6pt, short] (12.25,13) -- (12.75,13);
\draw [line width=0.6pt, short] (12.25,9.75) -- (12.5,9.5);
\draw [line width=0.6pt, short] (12.75,9.75) -- (12.5,9.5);
\draw [line width=0.6pt, short] (12.25,9.75) -- (12.75,9.75);
\end{circuitikz}
}%

\label{fig:my_label}
\end{figure}

\item %80
What is the change in the effective stress in kN/$m^2$ st mid-depth of the clay layer due to the lowering of the table?
\begin{enumerate}
\begin{multicols}{4}
\item 0
\item 20
\item 80
\item 100
\end{multicols}
\end{enumerate}

\item %81
What is the compression of the clay layer in mm due to the lowering of the water table?
\begin{enumerate}
\begin{multicols}{4}
\item 125
\item 100
\item 25
\item 0
\end{multicols}
\end{enumerate}

\textbf{Statement for Linked Answer Questions 82 and 83:}\\
A rectangular open channel needs to be designed to carry a flow of 2.0 $m^3$/s under uniform flow conditions. The Manning's roughness coefficient is 0.018. The channel should be such that the flow depth is equal to half the width,and the Froude number is equal to 0.5.
\item %82
The bed slope of the channel to be provided is
\begin{enumerate}
\begin{multicols}{4}
\item 0.0012
\item 0.0021
\item 0.0025
\item 0.0052
\end{multicols}
\end{enumerate}

\item %83
keeping the width, flow depth and roughness same, if the bed slope of the above channel is doubled, the average boundary shear stress under uniform flow conditions is 
\begin{enumerate}
\begin{multicols}{4}
\item 5.6 N/$m^2$
\item 10.8 N/$m^2$
\item 12.3 N/$m^2$
\item 17.2 N/$m^2$
\end{multicols}
\end{enumerate}

\textbf{Statement for Linked Answer Questions 84 and 85:}\\
A plain sedimentation tank with a length of 20 m, width of 10 m, and a depth of 3 mis used in a water treatment plant to treat 4 million litres of water per day \brak{4 MLD}. The average temperature of water is 20 \degree C. The dynamic viscosity of water is $1.002 \times 10^{-3}$ N.s /$m^2$ at 20 \degree C. Density of water is 998.2 kg/$m^3$. Average specific gravity of particles is 2.65.

\item %84
What is the surface overflow rate in the sedimentation tank?
\begin{enumerate}
\begin{multicols}{4}
\item 20 $m^3/m^2/day$
\item 40 $m^3/m^2/day$
\item 67 $m^3/m^2/day$
\item 133 $m^3/m^2/day$
\end{multicols}
\end{enumerate}




\end{enumerate}
\end{document}
