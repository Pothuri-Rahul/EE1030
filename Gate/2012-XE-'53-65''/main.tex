\let\negmedspace\undefined
\let\negthickspace\undefined
\documentclass[journal]{IEEEtran}
\usepackage[a5paper, margin=10mm, onecolumn]{geometry}
%\usepackage{lmodern} % Ensure lmodern is loaded for pdflatex
\usepackage{tfrupee} % Include tfrupee package
\setlength{\headheight}{1cm} % Set the height of the header box
\setlength{\headsep}{0mm}     % Set the distance between the header box and the top of the text
\usepackage{gvv-book}
\usepackage{gvv}
\usepackage{cite}
\usepackage{amsmath,amssymb,amsfonts,amsthm}
\usepackage{algorithmic}
\usepackage{graphicx}
\usepackage{textcomp}
\usepackage{xcolor}
\usepackage{tikz}
\usepackage{txfonts}
\usepackage{listings}
\usepackage{enumitem}
\usepackage{mathtools}
\usepackage{gensymb}
\usepackage{comment}
\usepackage[breaklinks=true]{hyperref}
\usepackage{tkz-euclide} 
\usepackage{listings}
% \usepackage{gvv}                                        
\def\inputGnumericTable{}                                 
\usepackage[latin1]{inputenc}                                
\usepackage{color}                                            
\usepackage{array}                                            
\usepackage{longtable}                                       
\usepackage{calc}                                             
\usepackage{multirow}                                         
\usepackage{hhline}                                           
\usepackage{ifthen}                                           
\usepackage{lscape}
\usepackage[a5paper, margin=10mm, onecolumn]{geometry}
\usepackage{graphicx}
\begin{document}
\bibliographystyle{IEEEtran}
\vspace{3cm}
\title{2012-XE-'53-65'}
\author{EE24BTECH11050 - Pothuri Rahul}
% \maketitle
% \newpage
% \bigskip
{\let\newpage\relax\maketitle}
\renewcommand{\thefigure}{\theenumi}
\renewcommand{\thetable}{\theenumi}
\setlength{\intextsep}{10pt} % Space between text and floats
\numberwithin{equation}{enumi}
\numberwithin{figure}{enumi}
\renewcommand{\thetable}{\theenumi}
Q. 10 - Q. 22 carry two marks each \\
\begin{enumerate}[start=10]
\item %10
Match the properties in Column I with the appropriate units in Column II \\
\begin{tabular}{l l}
\underline{Column I} & \underline{Column II} \\
P. Thermal diffusivity   & 1. $Hm^{-1}$ \\
Q. Fracture toughness    & 2. $m^2s^{-1}$ \\
R. Surface energy        & 3. $Fm^{-1}$ \\
S. Magnetic permeability & 4. $Nm^{-\frac{3}{2}}$ \\
                        & 5. $Jm^{-2}$
\end{tabular}
\begin{enumerate}
\begin{multicols}{2}
\item P-2,Q-5,R-4,S-1
\item P-2,Q-4,R-5,S-1
\item P-3,Q-4,R-5,S-3
\item P-5,Q-4,R-2,S-3
\end{multicols}
\end{enumerate}
\item %11
Match the characterization techniques in Column I with Column II \\
\begin{tabular}{l l}
\underline{Column I} & \underline{Column II} \\
P. Scanning tunneling microscopy & 1. No vacuum required \\
Q. Scanning electron microscopy & 2. Backscattered electrons \\
R. Transmission electro microscopy & 3. Photoelectrons \\
S. Atomic force microscopy & 4. Atomically sharp tip\\
& 5.  Sub-Angstrom resolution \\
\end{tabular}
\begin{multicols}{2}
    \begin{enumerate}
        \item P-4, Q-2, R-5, S-1
        \item P-1, Q-3, R-4, S-5
        \item P-2, Q-4, R-1, S-5
        \item P-5, Q-1, R-2, S-4
    \end{enumerate}
\end{multicols}
\item %12
Match the materials in Column I with the applications in Column II 
\begin{tabular}{l l}
\underline{Column I} & \underline{Column II} \\
P. Titanium diboride & 1. Photocatalyst \\
Q. Molybdenum disilicide & 2. Furnace heating element \\
R. Hydroxyapatite & 3. Ultra high temperature material \\
S. Nanocrystalline titanium oxide & 4. Tough ceramic \\
 & 5. Artificial bone implant \\
 \end{tabular}
\begin{enumerate}
\begin{multicols}{2}
\item P-3,Q-4,R-5,S-1
\item P-5,Q-3,R-2,S-1
\item P-4,Q-3,R-1,S-5
\item P-3,Q-2,R-5,S-1
\end{multicols}
\end{enumerate}
\item %13
Match the properties in column I with the options in Column II \\
\begin{tabular}{l l}
\underline{Column I} & \underline{Column II} \\
P. Toughness & 1. Resistance to plastic deformation \\
Q. Resilience & 2. Time dependent permanent deformation under constant load \\
R. Creep & 3. Total elongation at failure \\
S. Hardness & 4. Area under Stress-strain graph \\
 & 5. Area under the elastic part of the stress-strain curve \\
 \end{tabular}
 \begin{enumerate}
\begin{multicols}{2}
\item P-5,Q-1,R-3,S-2
\item P-4,Q-3,R-2,S-1
\item P-4,Q-5,R-2,S-1
\item P-5,Q-4,R-3,S-2
\end{multicols}
\end{enumerate}
\item %14
Determine the mole fraction of vinyl chloride in a copolymer of vinyl chloride \brak{CH_2CHCl} and vinyl acetate \brak{CH_2-CH-OCOCH_3} having molecular weight of $10520 g/mol$ and degree of polymerization of 160. 
\begin{enumerate}
\begin{multicols}{4}
\item 0.14
\item 0.30
\item 0.70 
\item 0.86
\end{multicols}
\end{enumerate}
\item %15
The electron concentration in an n-type semiconductor is $5 \times 10^{18}/m^3$. If the drift velocity of electrons is $100m/s$ in an electric fieldof $500V/m $, calculate the conductivity of the semiconductor.
\begin{enumerate}
\begin{multicols}{4}
\item $0.16 \times 10^{-1} S/m$
\item $1.6 \times 10^{-1} S/m$
\item $2.50 \times 10^{-1} S/m$
\item $30.05 \times 10^{-1} S/m$
\end{multicols}
\end{enumerate}
\item % 16
Calculate the saturation magnetization \brak{M_{sat}} for bcc iron of lattice parameter $2.866 \text{\AA} $. 
\begin{enumerate}
\begin{multicols}{4}
\item $0.76 \times 10^6 A/m$
\item $1.5 \times 10^6 A/m$
\item $3.15 \times 10^6 A/m$
\item $4.73 \times 10^6 A/m$
\end{multicols}
\end{enumerate}
\begin{center}
 COMMON DATA QUESTIONS
\end{center}
Common data questions for 17 and 18 :\\
A plain $0.45 wt.\% $ carbon steel is cooled slowly from $900 \degree C$ to just below the eutetoid temperature $\brak{723 \degree C}$ so that the following reaction occurs: 
\begin{center}
    $\gamma \brak{0.8 wt.\% c} \leftrightarrow \alpha \brak{0.02 wt.\% c} + Fe_3C\brak{6.67 wt.\% c}$
\end{center}
\item % 17 
During cooling from $900 \degree C$ to $700 \degree C$, the proeutectoid $\alpha$ forms from $\gamma$. Find the volume \% of proeutectoid $\alpha $ just below $723 \degree C$ for the steel.
\begin{enumerate}
\begin{multicols}{4}
\item $44.9 \%$
\item $66.1 \%$
\item $55.1 \%$
\item $34.9 \%$
\end{multicols}
\end{enumerate}
\item %18 
Find the volume \% of pearlite for the steel just below $723 \degree C$ for $0.45 \%$ carbon steel.
\begin{enumerate}
\begin{multicols}{4}
\item $44.9 \%$
\item $55.1 \%$
\item $40.9 \%$
\item $59.1\%$
\end{multicols}
\end{enumerate}
Common data questions for 19 and 20 :\\
A $20kN$ tensile load is applied axially to a steel bar of cross-section area $8 cm^2$ and $1m$ length. The Young's modulus of steel $\brak{E_{steel}}$ is $200GPa$ , and of aluminium $\brak{E_{Al}}$ is $70 GPa$. The Poisson's ratio \brak{v} can be taken as 0.3.
\item %19
When the same load is applied to an aluminium bar, it is found to give same elastic strain as the steel. Calculate the cross-section area of the aluminium bar.
\begin{enumerate}
\begin{multicols}{4}
\item $11.43 cm^2$
\item $14.93 cm^2$
\item $18.26 cm^2$
\item $22.86 cm^2$
\end{multicols}
\end{enumerate}
\item %20
Calculate the final area of the steel bar after the deformation under the applied load of $20kN$.
\begin{enumerate}
\begin{multicols}{4}
\item $7.9 cm^2$
\item $9.7 cm^2$
\item $7.0 cm^2$
\item $8.1 cm^2$
\end{multicols}
\end{enumerate}
Linked Answer Questions \\
Statement for  Linked answer questions 21 and 22 :\\
Chromium has the bcc structure with atomic diameter of $2.494 \text{\AA}$
\item %21
Calculate the lattice parameter of chromium assuming tight atomic bonding. 
\begin{enumerate}
\begin{multicols}{4}
\item $1.442 \text{\AA}$
\item $2.880 \text{\AA}$
\item $4.323 \text{\AA}$
\item $5.764 \text{\AA}$
\end{multicols}
\end{enumerate}
\item %22 
Find the first diffraction peak position $\brak{2 \theta}$ for Cu $K\alpha$ radiation with a wavelength of $1.54 \text{\AA}$
\begin{enumerate}
\begin{multicols}{4}
\item $21.76 \degree$
\item $33.05 \degree$
\item $44.43 \degree$
\item $66.10 \degree$
\end{multicols}
\end{enumerate}
\end{enumerate}
\end{document}
